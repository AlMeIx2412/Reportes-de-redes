\documentclass[12pt]{article}
\usepackage[utf8]{inputenc}
\usepackage[spanish]{babel}
\usepackage{url}
\usepackage{natbib}
\usepackage{graphicx}

\begin{document}
\begin{titlepage}
	\begin{center}
		
		% Upper part of the page. The '~' is needed because \\
		% only works if a paragraph has started.
		\noindent
		\begin{minipage}{0.5\textwidth}
			\begin{flushleft} \large
				\includegraphics[width=0.3\textwidth]{../../Images/ipn.png}
			\end{flushleft}
		\end{minipage}%
		\begin{minipage}{0.55\textwidth}
			\begin{flushright} \large
				\includegraphics[width=0.45\textwidth]{../../Images/logoescom.png}
			\end{flushright}
		\end{minipage}
		
		\textsc{\LARGE Instituto Politécnico Nacional}\\[0.5cm]
		
		\textsc{\Large Escuela Superior de Cómputo}\\[1cm]
		
		\textsc{\Large Redes de Computadoras}\\[1cm]
		
		% Title
		
		{ \huge Tarea 1. Medios de Transmisión\\[1cm] }
		
		{ \Large Grupo: 2CM12} \\[1cm] 
		\noindent
		\begin{minipage}{0.5\textwidth}
			\begin{flushleft} \large
				\emph{Integrantes:}\\
			     Mexicano Ixtepan Alejandro \\
			     Salinas Sierra Elizabeth \\
			\end{flushleft}
		\end{minipage}%
		\begin{minipage}{0.5\textwidth}
			\begin{flushright} \large
				\emph{Profesor:} \\
			    Moreno Cervantes Axel Ernesto
			\end{flushright}
		\end{minipage}
		
		\vfill
		
		
	\end{center}
\end{titlepage}


\tableofcontents

\section{Introducción}
Medios de transmisión.
Los medios de transmisión son las vías por las cuales se comunican los datos. Dependiendo de la forma de conducir la señal a través del medio o soporte físico, se pueden clasificar en dos grandes grupos:
***Medios de transmisión guiados.
***Medios de transmisión no guiados

\section{Medios de transmisión guiados}
Los medios de transmisión guiados están constituidos por cables que se encargan de la conducción (o guiado) de las señales desde un extremo al otro. Las principales características de los medios guiados son el tipo de conductor utilizado, la velocidad máxima de transmisión, las distancias máximas que puede ofrecer entre repetidores, la inmunidad frente a interferencias electromagnéticas, la facilidad de instalación y la capacidad de soportar diferentes tecnologías de nivel de enlace.              Dentro de los medios de transmisión guiados, los más utilizados en el campo de las telecomunicaciones y la ínter conexión de computadoras son tres:

\begin{center}
 \begin{tabular}{||p{3 cm} p{3 cm} p{3cm} p{3cm}||} 
 \hline
 \centering
 Nombre & Velocidad Maxima de Transmision & Ancho de Banda & Distancia entre repetidores\\ [3 ex] 
 \hline\hline
 Cable par trenzado & 1 GBPS & 250 KHZ  & 2-10 KM\\ 
 \hline
 Cable coaxial & 2 GBPS & 400 MKZ & 10-100 KM \\
 \hline
 Fibra óptica & >10 GBPS & 2 GHZ & >100KM \\
 \hline
 
\end{tabular}
\end{center}

\section{Medios de transmisión no guiados}
Los medios no guiados o comunicación sin cable transportan ondas electromagnéticas sin usar un conductor físico, sino que se radian a través del aire, por lo que están disponibles para cualquiera que tenga un dispositivo capaz de aceptarlas. En este tipo de medios tanto la transmisión como la recepción de información se lleva a cabo mediante antenas. 

La configuración para las transmisiones no guiadas puede ser direccional y omnidireccional. En la direccional, la antena transmisora emite la energía electromagnética concentrándola en un haz, por lo que las antenas emisora y receptora deben estar alineadas. En la omnidireccional, la radiación se hace de manera dispersa, emitiendo en todas direcciones, pudiendo la señal ser recibida por varias antenas. Generalmente, cuanto mayor es la frecuencia de la señal transmitida es más factible confinar la energía en un haz direccional.Según el rango de frecuencias de trabajo, las transmisiones no guiadas se pueden clasificar en tres tipos:

\begin{center}
 \begin{tabular}{||p{3 cm} p{3 cm} p{3cm} p{3cm}||} 
 \hline
 \centering
 Nombre & Velocidad Maxima de Transmision & Ancho de Banda & Distancia entre repetidores\\ [0.5ex] 
 \hline\hline
 Ondas de radio & 1 MBPS & 153 kHz a 30 MHz & 100-1000 KM\\ 
 \hline
 Microondas & 10 MBPS & 100 MHZ & 80 KM\\
 \hline
 Infrarrojo & 10 MBPS & 850-900 nm & 200 KM\\
 \hline
 Ondas luz & 1 MBPS & 322 THz & 1 KM\\
 \hline

\end{tabular}
\end{center}


\section{Conclusiones}

``I always thought something was fundamentally wrong with the universe'' \cite{adams1995hitchhiker}

\section{Bibliografias}
\begin{itemize}
\item {\url{https://sites.google.com/site/sabyrodriguezgamez/unidad1/1-3-medios-de-transmision}}
\item{\url{https://es.scribd.com/document/362876526/ Tabla-Comparativa-Medios-de-Transmision-Guiados-y-No-Guiados}}
\item {\url{https://anahi01blog.blogspot.com/}}
\item{\url{http://dis.um.es/~lopezquesada/documentos/IES_1213/LMSGI/curso/xhtml/xhtml6/tipos%20de%20cableado.html}}
\item {\url{https://www.ecured.cu/Medios_Guiados_y_no_Guiados}}
\item{\url{ https://sites.google.com/site/cableadoredpartrenzado/home/caracteristicas-tecnicas-del-par-trenzado}}
\end{itemize}
\end{document}
